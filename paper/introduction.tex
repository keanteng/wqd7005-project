\section{Introduction}

This academic report details an investigation into leveraging Artificial Intelligence (AI) for the early prediction of patient health deterioration, a critical aspect of modern healthcare aimed at improving outcomes, optimizing resource allocation, and reducing costs. Traditional monitoring often fails to capture subtle early decline, whereas AI, particularly Large Language Models (LLMs), Smaller Language Models (SLMs), and Generative AI (GenAI), offers transformative potential for continuous, proactive analysis of complex patient data. These models excel at understanding unstructured text like clinical notes (LLMs/SLMs) and generating synthetic datasets for research where real-world data is limited (GenAI). The primary objective of this project was to harness these AI capabilities for predictive modeling of patient deterioration. This was achieved by simulating a comprehensive patient dataset using GenAI, engineering relevant features from structured and unstructured data (including medical history and questionnaire responses via LLMs/SLMs), developing and evaluating machine learning and deep learning models for deterioration prediction, performing specialized NLP tasks for deeper textual insights, and utilizing AI to interpret complex model results. The scope of work thus encompassed dataset simulation, feature engineering, predictive modeling, thorough model evaluation, and AI-assisted reporting. This report is structured to first detail the methodology for these tasks (\textit{Section 2}), then present and discuss the results from classification and NLP tasks, including AI-assisted interpretations (\textit{Section 3}). Subsequently, \textit{Section 4} concludes with key findings and objective achievement, \textit{Section 5} offers recommendations for future work, and \textit{Section 6} discloses the AI tools and techniques employed.