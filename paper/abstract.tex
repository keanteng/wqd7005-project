This report details a project aimed at leveraging Artificial Intelligence (AI), including Large Language Models (LLMs), Small Language Models (SLMs), and Generative AI (GenAI), to predict patient health deterioration. The core objective was to explore the efficacy of these advanced technologies in a healthcare context, specifically through the simulation of patient vital signs and questionnaire data, and the subsequent development of predictive models. Key methods employed include dataset simulation using GenAI (\textit{Gemini 2.5 Flash-Preview-0417}), feature engineering from textual data via LLMs/SLMs (\textit{BioBERT, Gemini 2.5 Flash-Preview-0417}), development of both traditional machine learning (\textit{Random Forest, XGBoost, Neural Networks}) and transformer-based tabular models (\textit{TabTransformer, FT Transformer}) for deterioration prediction, and specialized Natural Language Processing (NLP) tasks such as sentiment analysis (\textit{Google BERT-Large}), clinical Named Entity Recognition (\textit{Bio-Clinical BERT}), and questionnaire response classification (\textit{BERT-base}). The main findings highlight the comparative performance of these diverse models, demonstrating strong predictive capabilities, particularly from the FT Transformer in patient deterioration and BERT-based models in NLP tasks. The effectiveness of AI in interpreting complex model outputs (using \textit{Gemini 2.5 Flash-Thinking-Preview-0417, Grok 3, Gemini 2.5 Flash-Thinking-Preview-0520}) was also a significant outcome. This work underscores the potential of advanced AI in enhancing predictive healthcare analytics and provides recommendations for future model refinement, data enhancement, and ethical deployment.