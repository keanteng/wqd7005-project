\section{Conclusions}

\subsection{Summary of Key Findings}

The investigation yielded several key findings regarding the application of AI in predicting patient deterioration and analyzing health-related textual data. For patient deterioration prediction, both traditional models, such as Random Forest, and advanced transformer-based models like FT Transformer, achieved exceptionally high performance, with Area Under the Curve (AUC) scores exceeding 0.99. Random Forest demonstrated a strong balance in its predictive capabilities, while FT Transformer particularly excelled in achieving the highest AUC. Specialized Natural Language Processing (NLP) tasks utilizing BERT variants also proved successful. Sentiment analysis performed on lifestyle descriptions yielded good results, achieving an AUC of 0.92 for Fatigue and 0.89 for other categories. Named Entity Recognition (NER) on medical history, using Bio-Clinical BERT, attained a high F1-score of 0.926 and near-perfect token-level AUC. Furthermore, the classification of questionnaire responses into risk categories performed well, most notably for Fatigue, which registered an AUC of 0.93. Generative AI, specifically Gemini, was found to be effective for simulating the initial patient dataset and for feature engineering by scoring textual content. AI tools also proved valuable in interpreting model results, providing nuanced analyses of metrics and visualizations. A significant challenge identified across several tasks was the impact of small dataset sizes, which introduced risks of overfitting and potential limitations in generalizing findings to new, unseen data.

\subsection{Achievement of Objectives}

The project successfully met its stated objectives. This was achieved through the effective leveraging of Generative AI for dataset simulation, addressing initial data scarcity. Large Language Models (LLMs) and Small Language Models (SLMs) were successfully employed for feature engineering, extracting valuable insights from textual data. A range of machine learning and deep learning models were developed and rigorously evaluated for their efficacy in patient deterioration prediction. Furthermore, the project successfully performed and evaluated specialized NLP tasks, including sentiment analysis, Named Entity Recognition, and text classification. Finally, AI was utilized to facilitate the interpretation of complex model results, enhancing the understanding of model performance and behavior.

\subsection{Significance of the Work}

This project demonstrates the significant potential of integrating advanced AI and NLP techniques into healthcare analytics, particularly for predicting patient health deterioration. The ability to extract meaningful information from unstructured patient data, such as medical history and questionnaire responses, and combine it with structured vital signs for predictive modeling can lead to more accurate and timely interventions. The use of GenAI for data simulation offers a valuable pathway for research and development, especially in scenarios where access to real-world patient data is limited or restricted. Furthermore, the application of AI for results interpretation can accelerate the insight generation process, making the behaviors of complex models more accessible and understandable to healthcare professionals. While the findings are promising, they also underscore the critical need for sufficient data and rigorous validation processes when deploying AI solutions in the sensitive domain of healthcare.